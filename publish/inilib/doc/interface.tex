% -*- latex -*-
%
This software may only be used by you under license from the
University of Notre Dame.  A copy of the University of Notre Dame's
Source Code Agreement is available at the inilib Internet website
having the URL: <http://inilib.sourceforge.net/license/> If you
received this software without first entering into a license with the
University of Notre Dame, you have an infringing copy of this software
and cannot use it without violating the University of Notre Dame's
intellectual property rights.
% 
% $Id: interface.tex,v 1.7 2000/09/04 01:20:46 bwbarrett Exp $
%

\section{Interface}

This section describes the interface provided by {\tt inilib}.

Many of the methods provided in the {\tt inilib} package take one of
four data types ({\tt bool}, {\tt double}, {\tt int} and {\tt
std::string}) as an argument.  For example, the {\tt section.insert()}
has four prototypes:

\vspace{11pt}
\noindent {\tt void insert(const std::string\& key, bool value);} \\
\noindent {\tt void insert(const std::string\& key, double value);} \\
\noindent {\tt void insert(const std::string\& key, int value);} \\
\noindent {\tt void insert(const std::string\& key, std::string value);} \\

For clarity, since {\tt inilib} frequently provides overloaded
functions for use with any of the four supported types, the type will
be denoted by the {\sc Smallcaps} font, which denotes one of the four
({\tt int}, {\tt double}, {\tt std::string} and {\tt bool}) supported
types:

\vspace{11pt}
\noindent {\tt void insert(const std::string\& key, {\sc Datatype} value);} \\

The term ``target'' is used below to mean the object upon which the
member function is invoked upon.  The term ``source'' typically refers
to the argument of the member function.

Additionally, the term ``deep copy'' is used to denote a copy where
all data that is included in an object is copied to the target.  This
means that after the copy, there are two distinct copies of the data
(as opposed to two objects that refer to the same underlying data).

\subsection{Namespace}

All classes, methods, and operators in the {\tt inilib} library are
contained in the {\tt INI} namespace.

% From here, the subsections for the parts are included from inilib.tex
