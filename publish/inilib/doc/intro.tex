% -*- latex -*-
%
This software may only be used by you under license from the
University of Notre Dame.  A copy of the University of Notre Dame's
Source Code Agreement is available at the inilib Internet website
having the URL: <http://inilib.sourceforge.net/license/> If you
received this software without first entering into a license with the
University of Notre Dame, you have an infringing copy of this software
and cannot use it without violating the University of Notre Dame's
intellectual property rights.
% 
% $Id: intro.tex,v 1.14 2000/09/04 01:20:46 bwbarrett Exp $
%

\section{Introduction}
\label{sec:Introduction}

{\tt inilib} is a C++\cite{cpp98} library which provides a convenient
mechanism for saving the ``state'' of a program in the well-known
``{\tt .ini} file'' format used in Microsoft
Windows$^{TM}$~\cite{MS-WIN}.  {\tt inilib} gives the programmer a
means of storing a number of arbitrary settings in memory with an easy
access interface, as well providing means for saving the information
to and loading it from a file on disk.  Data is stored in an easy to
read format, allowing the user to modify any of the information with a
simple text editor (i.e., outside the scope of {\tt inilib}).

{\tt inilib} benefits the programmer by providing a simple, intuitive
means to store any data that can be expressed in {\tt std::string},
{\tt int}, {\tt double}, or {\tt bool} types.  In addition, {\tt
  inilib} handles any conversions that may be necessary to convert
from one type to the other.  Saving all information stored in {\tt
  inilib} to disk or loading information from disk requires only one
function call.

{\tt inilib} provides a hierarchy for access of information, in order
to provide the most flexibility for programmers.  The top level is
called the {\tt registry}, where each registry corresponds to one
file.  Inside of a {\tt registry}, there are zero or more {\tt
  section}s.  A {\tt section} contains zero or more {\tt attribute}s,
each of which actually store the information.  Each section is
intended to contain all the information for a particular topic.  For
example, a text editor program might have a section for window
location and size, and another section containing the last documents
opened by the user.  Figure~\ref{fig:sample_file} shows a possible
{\tt .ini} file for such a program.

\begin{figure}[!h]
\mybox{[window settings] \\
width = 400 \\
height = 500 \\
xpos = 10 \\
ypos = 5 \\

[history] \\
file1 = /home/bbarrett/document1.txt \\
file2 = /home/bbarrett/document2.txt \\
file3 = /home/bbarrett/document3.txt }
\caption{An example file produced by {\tt inilib}.}
\label{fig:sample_file}
\end{figure}

Using the information provided in Figure~\ref{fig:sample_file} is
simple.  An example implementation that prints out the last file
opened (assuming that {\tt file1} always points to the last file
opened) is shown in Figure~\ref{fig:print_file}.  The {\tt registry}
constructor loads the file {\tt sample.ini}, from
Figure~\ref{fig:sample_file}.  The next line demonstrates accessing a
piece of information from {\tt inilib}.  The string in the first {\tt
  []} gives the {\tt section} to access, 
and the string in the second {\tt []} specifies which {\tt attribute}
to return.  

\begin{figure}[!h]
\mybox{\#include <iostream> \\
\#include <string> \\
\#include "inilib.h" \\
\\
using namespace std; \\
\\
int \\
main(int argc, char *argv[]) \\
$\{$ \\
\hspace*{1em}// Create the object and read the file \\
\hspace*{1em}INI::registry information("sample.ini"); \\
\\
\hspace*{1em}// Output a given attribute \\
\hspace*{1em}cout << information["history"]["file1"] << endl; \\
\\
\hspace*{1em}return 0; \\
$\}$ }
\caption{Printing out the last file ({\tt file1}) opened in the
example text editor.}
\label{fig:print_file}
\end{figure}
