% -*- latex -*-
%
This software may only be used by you under license from the
University of Notre Dame.  A copy of the University of Notre Dame's
Source Code Agreement is available at the inilib Internet website
having the URL: <http://inilib.sourceforge.net/license/> If you
received this software without first entering into a license with the
University of Notre Dame, you have an infringing copy of this software
and cannot use it without violating the University of Notre Dame's
intellectual property rights.
% 
% $Id: registry.tex,v 1.15 2001/04/17 22:05:45 jsquyres Exp $
%

\subsection[registry Class]{{\tt registry} Class}
\label{sec:registry}

As described in Section~\ref{sec:overview}, {\tt inilib} is designed
such that all access to data must begin with an instance of the
registry class.  

The {\tt registry} class is directly associated with a filename, which
will be used to populate the {\tt registry} or save the information in
the {\tt registry} to disk.  Constructors are provided to populate a
{\tt registry} immediately upon creation.  In addition, it is possible
to have a {\tt registry} write itself to disk when its destructor is
called.

All of the member functions in the {\tt registry} class are summarized
in Table~\ref{tbl:registry-members}.

\begin{table}[htbp]
  \begin{center}
    \leavevmode
    \begin{tabular}{|l|p{3in}|}
      \hline
      \multicolumn{1}{|c|}{Name} & \multicolumn{1}{c|}{Purpose} \\
      \hline
      Default constructor & Create an empty registry. \\
      Copy constructor & Perform a deep copy. \\
      Other constructor & Associate a filename with the registry, and
      optionally set read-upon-construction and write-upon-destruction
      flags. \\
      Destructor & Destroy the registry and all associated data. \\
      \hline
      {\tt operator=} & Assignment operator; clear the registry and
      perform deep copy.\\
      {\tt operator+=} & Perform a deep copy/append. \\
      {\tt operator[]} & Return a {\tt section}. \\
      {\tt insert} & Insert a new {\tt section}. \\
      {\tt clear} & Erase all data in the registry. \\
      {\tt find} & Return an iterator to a {\tt section}. \\
      {\tt empty} & Return {\tt true} if there are no sections, {\tt
        false} otherwise. \\
      {\tt begin} & Return iterator to beginning of the registry. \\
      {\tt end} & Return iterator past the end of the registry. \\
      {\tt file\_read} & Read in a specified (or implied) {\tt .ini}
      file. \\ 
      {\tt file\_write} & Write out a specific (or implied) {\tt
        .ini} file. \\
      {\tt set\_filename} & Associate a filename with the registry. \\
      {\tt get\_filename} & Get the filename that is associated with
      the registry. \\
      {\tt set\_write\_on\_destruct} & Set whether the registry will
      be written out upon destruction. \\
      {\tt get\_write\_on\_destruct} & Get whether the registry will
      be written out upon destruction. \\
      \hline
    \end{tabular}
    \caption{Member functions in the {\tt registry} class.}
    \label{tbl:registry-members}
  \end{center}
\end{table}

\subsubsection{Convenience typedefs}

Since the {\tt registry} class stores its sections in an STL {\tt
  map}, iterator access is provided to traverse all the sections in a
{\tt registry}.  To that end, the following typedefs are provided by
the {\tt registry} class:

\begin{itemize}
\item {\tt typedef std::map<std::string, section\&>::iterator iterator;}
\item {\tt typedef std::map<std::string, section\&>::const\_iterator
const\_iterator}
\end{itemize}

These typedefs can be used similar to STL iterators.  An example is
given in Figure~\ref{fig:iterator}.

\begin{figure}[!h]
\mybox{\#include <iostream> \\
\#include <string> \\
\#include "inilib.h" \\
\\
using namespace std; \\
\\
int \\
main(int argc, char *argv[]) \\
\{ \\
\hspace*{1em}INI::registry reg("user.ini"); \\
\hspace*{1em}// Populate the registry \\
\hspace*{1em}reg["window settings"]["width"] = 300; \\
\hspace*{1em}reg["window settings"]["height"] = 300; \\
\hspace*{1em}reg["history"]["file1"] = ""; \\
\hspace*{1em}reg["history"]["file2"] = ""; \\
\hspace*{1em}reg["history"]["file3"] = ""; \\
\\
\hspace*{1em}// print out the name of every section in the \\
\hspace*{1em}// registry. \\
\hspace*{1em}for (registry::iterator i = reg.begin() ;  \\
\hspace*{3.5em}i != reg.end() ; \\
\hspace*{3.5em}++i) \\
\hspace*{2em}cout << (*i).first << endl; \\
\\
\hspace*{1em}return 0; \\
\} }
\caption{Using the iterator typedefs provided by the registry class.}
\label{fig:iterator}
\end{figure}
 

\subsubsection{Constructors}
\label{sec:registry:const}

\bind{registry()}
\bind{registry(const registry\&)}
\bind{registry(const std::string\& file, bool fileread = true, bool
  filewrite = true)}

Creates an instance of the {\tt registry} class.  {\tt file} is a
filename to associate with the instance of {\tt registry}.  If {\tt
  fileread} is true, the constructor will call {\tt file\_read()}.  If
{\tt filewrite} is true, the contents of the target {\tt registry}
will be written to {\tt file} when the destructor is called
(Section~\ref{sec:registry:dest}).  The associated filename can be
obtained with the {\tt get\_filename()} method
(Section~\ref{sec:registry:get_filename}) and changed with the {\tt
  set\_filename()} method (Section~\ref{sec:registry:set_filename}).
The behavior of the write on destruct option can also be controlled
with {\tt get\_write\_on\_destruct()}
(Section~\ref{sec:registry:get_write}) and {\tt
  set\_write\_on\_destruct()} methods
(Section~\ref{sec:registry:set_write}).

\subsubsection{Destructor}
\label{sec:registry:dest}

\bind{\~{}registry()}

Eliminates the instance of the {\tt registry} class.  All {\tt
section}s and {\tt attribute}s associated with the particular {\tt
registry} are destroyed as well.  If {\tt get\_write\_on\_destruct()}
returns true, the contents of the target {\tt registry} will be saved
to disk before the instance is deleted.

If the write fails, no notification will be given to the user.


\subsubsection{Assignment Operator}
        
\bind{registry\& operator=(const registry\& r)}

Assigns {\tt r} to the target registry.  This function first deletes
all information in the target registry, and makes a deep copy of the
information from {\tt r}, including the value of the {\tt
write\_on\_destruct} tag, into the registry.  

A reference to the target {\tt registry} is returned.


\subsubsection[operator+=]{{\tt operator+=}}
\label{code:reg:op_pl_eq}

\bind{registry\& operator+=(const registry\& r)}

Adds the information from {\tt r} to the target registry.  If a
section in {\tt r} does not exist in the target registry, the section
will be deep copied into the target (including all attributes in {\tt
r}).  The value of the {\tt write\_on\_destruct} tag will not be
copied from {\tt r}.

If a section in {\tt r} already exists in the target registry, all
attributes from the section in {\tt r} will be deep copied into the
section in the registry.  If an attribute in the registry exists
already, the attribute in {\tt r} will overwrite it.

A reference to the target registry is returned.


\subsubsection{{\tt operator[]}}

\bind{section\& operator[](const std::string\& key)}

Returns section referenced by key {\tt key} from the registry.  If the key
{\tt key} does not already exist in the registry, an empty section is
created, stored in the registry (which can be accessed in the future
by the key {\tt key}), and a reference to it is returned.


\subsubsection[insert]{{\tt insert}}

\bind{void insert(const std::string\& key, const section\& sec)}
\bind{void insert(const registry\& r)}

Inserts information into the target registry by deep copying the
source information.

If {\tt r} is passed to the function, {\tt insert} works in the same
way as {\tt operator+=} (see Section~\ref{code:reg:op_pl_eq}).

If {\tt key}, {\tt sec} are passed to the function, {\tt insert} adds
the section's information to the target registry.  If {\tt key} does
not exist in the registry, it will be created.  If it does exist, all
attributes will be copied into the existing section.  If a section
already exists with key {\tt key}, {\tt section::operator+=}
(Section~\ref{code:sec:op_pl_eq}) is used to combine the two sections.


\subsubsection[clear]{{\tt clear}}

\bind{void clear()}

Removes all information from the registry.  All sections (and all
attributes in each section) are deleted.


\subsubsection[find]{{\tt find}}
\label{code:reg:find}

\bind{iterator find(const std::string\& key)}

Returns an iterator pointing to a (name, section) pair with name {\tt
  key}.
%
The section name is the {\tt first} element of the pair, and is a
{\tt std::string}; the {\tt second} element of the pair is a reference
to the {\tt section} of that name.
%
If there is no section in the registry with the name {\tt key}, an
iterator equal to {\tt registry::end()} (see
Section~\ref{code:reg:end}) is returned.


\subsubsection[empty]{{\tt empty}}

\bind{bool empty()}

Returns {\tt true} if there are no sections in the target registry,
{\tt false} otherwise.


\subsubsection[begin]{{\tt begin}}

\bind{iterator begin()}

Returns the iterator returned by the underlying {\tt std::map}'s {\tt
begin()} function.  
%
The iterator points to (name, section) pairs, as described in the {\tt
  find()} function (Section~\ref{code:reg:find}).


\subsubsection[end]{{\tt end}}
\label{code:reg:end}

\bind{iterator end()}

Returns the iterator returned by the underlying {\tt std::map}'s {\tt
end()} function.


\subsubsection[file\_read]{{\tt file\_read}}

\bind{bool file\_read()}
\bind{bool file\_read(const std::string\& filename)}

If no argument is provided, attempts to read the file associated with
the target registry, either through the constructors
(Section~\ref{sec:registry:const}) or {\tt set\_filename()}
(Section~\ref{sec:registry:set_filename}).

If an argument is provided, populates the registry with data found in
the specified file.

In both cases, information from the file is {\em appended} to the
target registry.  Any attributes contained in both the registry and
the file will be overwritten with the values from the file.  If a
section is not already in the target registry, it will be added to the
target registry.  Likewise, if an attribute is not already in the
registry, it will be added.  See Section~\ref{sec:overview} for an in-depth
explanation of the underlying implementation of attributes.

Returns {\tt true} on success, {\tt false} otherwise.

\subsubsection[file\_write]{{\tt file\_write}}

\bind{bool file\_write()}
\bind{bool file\_write(const std::string\& filename)}

If no argument is provided, attempt to write to the filename
associated with the target registry (either through the constructors
(Section~\ref{sec:registry:const}) or {\tt set\_filename()}
(Section~\ref{sec:registry:set_filename}).

If the filename argument is provided, output the data contained in the
registry to the specified file.  Any information already in the file
will be overwritten.

Returns true on success, false otherwise.

\subsubsection[set\_filename]{{\tt set\_filename}}
\label{sec:registry:set_filename}

\bind{void set\_filename(const std::string\& name)}

Changes the filename associated with the target registry to {\tt name}.


\subsubsection[get\_filename]{{\tt get\_filename}}
\label{sec:registry:get_filename}

\bind{std::string get\_filename()}

Returns a string containing the filename currently associated with the
target registry.


\subsubsection[set\_write\_on\_destruct]{{\tt set\_write\_on\_destruct()}}
\label{sec:registry:set_write}
void \bind{set\_write\_on\_destruct(bool value)}

If {\tt value} is true, the target registry will attempt to write
itself to the file associated with it upon destruction.  If {\tt
value} is false, it will not attempt to write itself on destruction.


\subsubsection[get\_write\_on\_destruct]{{\tt get\_write\_on\_destruct()}}
\label{sec:registry:get_write}

\bind{bool get\_write\_on\_destruct()}

Returns {\tt true} if the target registry will attempt to write itself
to the file associated with it upon destruction.  Returns {\tt false}
otherwise.
