% -*- latex -*-
%
This software may only be used by you under license from the
University of Notre Dame.  A copy of the University of Notre Dame's
Source Code Agreement is available at the inilib Internet website
having the URL: <http://inilib.sourceforge.net/license/> If you
received this software without first entering into a license with the
University of Notre Dame, you have an infringing copy of this software
and cannot use it without violating the University of Notre Dame's
intellectual property rights.
% 
% $Id: section.tex,v 1.8 2001/04/17 22:05:45 jsquyres Exp $
%

\subsection[section class]{{\tt section} class}

The {\tt section} class is designed to be contained in a registry.  It
contains attributes, which are the actual data stored in a registry.
Table~\ref{tbl:section-members} summarizes the function members of the
{\tt section} class.

\begin{table}[htbp]
  \begin{center}
    \leavevmode
    \begin{tabular}{|l|p{3in}|}
      \hline
      \multicolumn{1}{|c|}{Name} & \multicolumn{1}{c|}{Purpose} \\
      \hline
      Default constructor & Create an empty section. \\
      Copy constructor & Perform a deep copy. \\
      Destructor & Destroy the section and all associated data. \\
      \hline
      {\tt operator=} & Assignment operator; clear the section and
      perform deep copy.\\
      {\tt operator+=} & Perform a deep copy/append. \\
      {\tt operator[]} & Return a {\tt section}. \\
      {\tt insert} & Insert a new {\tt section}. \\
      {\tt clear} & Erase all data in the section. \\
      {\tt find} & Return an iterator to a {\tt attribute}. \\
      {\tt empty} & Return {\tt true} if there are no attributes, {\tt
        false} otherwise. \\
      {\tt begin} & Return iterator to beginning of the section. \\
      {\tt end} & Return iterator past the end of the section. \\
      \hline
    \end{tabular}
    \caption{Member functions in the {\tt section} class.}
    \label{tbl:section-members}
  \end{center}
\end{table}

\subsubsection{Convenience typedefs}

Since the section class stores its attributes in an STL {\tt map},
iterator access is provided to traverse all the attributes in a
section.  To that end, the following typedefs are provided by the {\tt
section} class:

\bind{std::map<std::string, attribute\&>::iterator iterator}
\bind{std::map<std::string, attribute\&>::const\_iterator const\_iterator}

\subsubsection{Constructors}

\bind{section()}
\bind{section(const section\& s)}

Creates an instance of the {\tt section} class.  If a section {\tt s}
is given, {\tt s} is deep copied (along with all attributes in {\tt
s}) into the target section.

\subsubsection{Destructor}

\bind{\~{}section()}

Eliminates the instance of the {\tt section} class.  All {\tt
attribute}s in the section will be destroyed as well.

\subsubsection{Assignment Operator}

\bind{section\& operator=(const section\& s)}

Deep copies section {\tt s}.  All attributes in the target section
before calling {\tt operator=} are deleted from the section.  Copies
of all attributes in section {\tt s} are added to the target section.


\subsubsection[operator+=]{{\tt operator+=}}
\label{code:sec:op_pl_eq}

\bind{section\& operator+=(const section\& s)}

Deep copies section {\tt s}.  All attributes in the section before
calling {\tt operator+=} are kept in the section.  In the event that
there is an attribute in {\tt s} with the same key as an attribute in
{\tt this}, the attribute in {\tt s} will overwrite the current
attribute.


\subsubsection{{\tt operator[]}}

\bind{attribute\& operator[](const std::string\& key)}

Returns the attribute with key {\tt key}.  If no such attribute exists
in the section, a default attribute is returned (see
Section~\ref{sec:implementation:attribute}).


\subsubsection[insert]{{\tt insert}}

\bind{void insert(const std::string\& key, {\sc Datatype} value)}
\bind{void insert(const section\& sec)}

When {\tt insert} is called with parameter {\tt sec}, {\tt insert}
acts like {\tt operator+=} (see Section~\ref{code:sec:op_pl_eq}).

When {\tt insert} is called with parameters {\tt key} and {\tt value},
the attribute is added to the target section.  If an attribute with
key {\tt key} already exists, it is overwritten.


\subsubsection[clear]{{\tt clear}}

\bind{void clear()}

Removes all information from the {\tt section}.  All {\tt attribute}s
are deleted.


\subsubsection[find]{{\tt find}}
\label{code:sec:find}

\bind{iterator find(const std::string\& key)}

Returns an iterator pointing to the (name, attribute) pair with name
{\tt key}.
%
The attribute name is the {\tt first} element of the pair, and is a
{\tt std::string}; the {\tt second} element of the pair is a reference
to the {\tt attribute} of that name.
%
If there is no attribute in the section with the name {\tt key}, an
iterator equal to {\tt section::end()} (see
Section~\ref{code:sec:end}) is returned.


\subsubsection[empty]{{\tt empty}}

\bind{bool empty()}

Returns {\tt true} if there are no {\tt attribute}s in the target
section.  Returns {\tt false} otherwise.


\subsubsection[begin]{{\tt begin}}

\bind{iterator begin()}

Returns the iterator returned by the underlying {\tt std::map}'s {\tt
begin()} function.
%
The iterator points to (name, attribute) pairs, as described in the
{\tt find()} function (Section~\ref{code:reg:find}).

\subsubsection[end]{{\tt end}}
\label{code:sec:end}

\bind{iterator end()}

Returns the iterator returned by the underlying {\tt std::map}'s {\tt
end()} function.


