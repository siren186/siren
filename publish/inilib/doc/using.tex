% -*- latex -*-
%
This software may only be used by you under license from the
University of Notre Dame.  A copy of the University of Notre Dame's
Source Code Agreement is available at the inilib Internet website
having the URL: <http://inilib.sourceforge.net/license/> If you
received this software without first entering into a license with the
University of Notre Dame, you have an infringing copy of this software
and cannot use it without violating the University of Notre Dame's
intellectual property rights.
% 
% $Id: using.tex,v 1.7 2000/09/04 01:20:46 bwbarrett Exp $
%

\section[Using inilib]{Using {\tt inilib}}

This section contains information on obtaining {\tt inilib},
installing it, and getting help if something goes wrong.  In addition,
this section describes the coding standards maintained in the creation
of {\tt inilib}.

\subsection[Obtaining inilib]{Obtaining {\tt inilib}}
The source code for {\tt inilib} is available from the project's
homepage:

\begin{center}
{\tt http://inilib.sourceforge.net/}
\end{center}

Packages containing the official, supported releases are available
from the web page, as well as CVS access for the most current code.
The CVS code -- unless otherwise noted -- should be considered
unstable, and may not function as intended, and may therefore break
your applications.

The official {\tt inilib} distribution contains the source code in
{\tt src/}, as well as a comprehensive test suite that can be found at
{\tt contrib/test\_suite/} and usage examples in {\tt
contrib/examples/}.  The test suite is discussed in
Section~\ref{sec:codestd}.  The usage examples contain extensive
comments on the usage of {\tt inilib}.  In addition, the documentation
for {\tt inilib} is available in the {\tt doc/} directory of the
distribution and on the project's home page.

\subsection[Installing inilib]{Installing {\tt inilib}}

{\tt inilib} uses the GNU {\tt autoconf} and {\tt automake} utilities
to create a build process that works across all tested platforms.  In
addition, it may work on platforms that are not tested or supported.
For most cases, the build process is as simple as:

\begin{verbatim}
% ./configure
% make
% make examples
% cd contrib/test_suite ; ./inilib_test ; cd ../..
% make install
\end{verbatim}

The {\tt make examples} and {\tt ./inilib\_test} steps are optional,
but it is recommended you use them.  For file access reasons, you must
be in the same directory as the {\tt inilib\_test} binary and {\tt
test.ini} file in order to run the test suite.

C++ does not enjoy the same standard methodology of building static
libraries across different platforms and compilers like C does.
Indeed, for C libraries, the use of {\tt ar}(1) and (sometimes) {\tt
  ranlib}(1) is all that is required.  However, C++ functions
(particularly where templates are involved) may require multiple
passes from the compiler before a usable library can be produced.
There is currently no uniform manner to produce C++ libraries across
platforms/compilers.   It is hoped that the next release of the GNU
{\tt libtool} project will address this issue.  Until then, only the
platforms and compilers listed in Table~\ref{tbl:supported} are
supported for use with inilib.  

If your compiler is not listed, it is quite possible that your it will
work properly with inilib -- it should be a fairly straightforward
task to modify {\tt configure.in} for the right library compilation
hooks for your C++ compiler.

\subsection{Getting Help}

Thanks to SourceForge, bug tracking and reporting software and mailing
lists are available from the {\tt inilib} homepage.  The authors of
{\tt inilib} are Brian Barrett and Jeff Squyres.  To contact the
authors about a problem with {\tt inilib}, please use the support
listserv, {\tt inilib-support@lists.sourceforge.net}.  

There is also a development list, {\tt
inlib-devel@lists.sourceforge.net}.

 
\subsection{Coding Standards}
\label{sec:codestd}

{\tt inilib} was developed under a fairly stringent set of standards
that should ensure proper functionality.

The primary development environment for {\tt inilib} is Sun Solaris
2.6 (UltraSparc) using the Sun Workshop 5.0 compilers.  All releases
are verified to be free of memory leaks,\footnote{Free from avoidable
  memory leaks.  The Workshop 5.0 STL implementation has a few small
  memory leaks in the iostream implementation, so {\tt bcheck} will
  find some memory leaks in {\tt inilib}.} using {\tt bcheck}, a
memory checker that is part of the Workshop compiler suite.

In addition to being free of memory leaks, {\tt inilib} releases will
compile without warnings on all supported platforms using all
supported compilers.  GNU extenstions and non-standard extensions to
the STL will not be used, to increase portability.

A comprehensive test suite is included with the {\tt inilib} source.
It can be found in the {\tt contrib/test\_suite} directory.  This test
suite is intended to test the entire {\tt inilib} product for proper
functionality.  All releases canidates will be verified using the test
suite on all supported platform/compiler combinations (See
Section~\ref{sec:supported}) before being released.

\subsection{Supported Platforms}
\label{sec:supported}

Table~\ref{tbl:supported} shows the platform/compiler combinations
have been tested and are supported by {\tt inilib}.  Other platforms
may work, but may require some modifications to the build script.
{\tt g++ 2.95.2} may provide the easiest porting, as its C++ library
build process is the same as the standard C library build process.

\begin{table}
  \begin{center}
    \begin{tabular}[hb!]{| l | l |} 
      \hline
      \multicolumn{1}{|c|}{Platform} & \multicolumn{1}{c|}{Compiler}
      \\
      \hline
      Solaris 2.6 (Sparc)    & Workshop 5.0 CC \\
      & KCC 3.4f \\
      & g++ 2.95.2 \\ \hline
      Solaris 7 (Sparc)      & Workshop 5.0 CC \\
      & KCC 3.4f \\
      & g++ 2.95.2 \\ \hline
      x86 Linux (RedHat 6.2) & KCC 3.4f \\
      & g++ 2.95.2 \\ \hline
      MIPS Irix              & Irix CC \\
      & KCC 3.4f \\
      & g++ 2.95.2 \\ \hline
    \end{tabular}
  \end{center}
  \caption{Platform / compiler combinations supported by {\tt inilib.}}
  \label{tbl:supported}
\end{table}


